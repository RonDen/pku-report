\textbf{摘要}:“二战”以后,美国迅速成为世界政治、军事和经济的头号强国,科学技术在其背后的推动作用,功不可没。《科学-没有止境的前沿》这本由布什编写的政府工作报告,对美国科技政策的制定,以及之后的科学技术繁荣发展,起到了无可替代的重要作用。这本书展现了科学的前景——作为“没有止境的边疆”的科学将会取代美国西部物理上的边疆,成为国家的经济发展、提高生活标准和推动社会变化的新的动力。该报告回应了罗斯福总统提出的有关美国战后科学发展的四大问题,扭转了美国科学发展“重应用,轻基础”的历史取向,它是战后美国科学发展的道路上的重大历史拐点,具有重大历史地位和深远启迪意义。其中提出的许多观点,例如重视基础研究,基础研究是产生知识的科学;保证科学的自主性等,在现在看来,也是很有价值和指导意义的。

\textbf{关键词}:科技政策,创新,基础研究,布什,国家科学基金

科技政策研究的三个核心问题是国家科学技术政策的制定,科学技术资源的分配和对科学技术的规范和控制。万尼瓦尔·布什编写的《科学,没有止境的前沿》这份报告,是科技政策研究领域的经典之作,奠定了国家支持科学和教育发展的思想基础,标志着国家科技政策的正式出现,对美国及大多数发达国家科技政策的影响达半个世纪。

目前我国处于向高质量发展转型的关键期、机遇期,研究科技政策,思考政府与科学共同体之间协作的关系,对于思考我国现在科技政策与体制上存在的问题(例如李衍达院士在文献\cite{ref3}讨论的科技与经济“两张皮”现象)、激发科技创新、促进科技发展等具有重要的意义。

“二战”前的美国,对科学技术的重视程度不够,走的是一条“重应用轻基础”的道路\cite{ref2}。
“二战”期间以及随后的冷战期间产生的一系列最前沿的新技术成为促进美国经济增长的高技术:电子计算机、商用运输机、半导体、固体电子仪器、集成电路、核能、激光、卫星通讯、微波通讯、雷达的应用(例如导航控制)、抗生素、杀虫剂、新材料(例如高强度铁合金、钛、高温陶瓷、光纤强化塑料、复合材料)、金属制造和加工的新方法(例如数控机床)以及今天的互联网。而在“二战”刚爆发的时候,按诺贝尔奖获奖者的数目来说,美国还远远落后于德国、与英国也相差很远、也落后于法国。再早十多年,直到20世纪30年代早期希特勒掌权的时候,美国最聪明、最有抱负的年轻人都会远到德国像海德堡、莱比锡和哥根廷这样城市中的大学去攻读博士学位。是什么使美国的科学技术实力一跃直上呢?目前我国也处在建设社会主义强国,提高自主创新能力的关键发展机遇窗口期,研究这个问题,必然是很有意义的。带着这份强烈的求知欲,我读完了《科学:没有止境的前沿》这本书。

\section{报告背景与由来}

在第二次世界大战期间,美国动员了其充足的科技资源投入到备战之中。1940年,罗斯福总统建立了美国国防研究委员会,并委任时任卡耐基研究院的院长布什担任该委员会的主席。该委员会的职责是组织科技资源来增强国防能力。为完成这一使命,需要新的管理机构。1941年罗斯福总统下令建立科学研究与发展局(OSRD),布什担任主任。该办公室自身并不从事研发,它以合同的形式,利用全国的大学、产业和政府研究机构建立联系,开展国防研究,以尽早结束战争。

到1944年,战争就要结束了。罗斯福总统请求布什总结战时国防研究的经验教训,
看看它们能否对增进美国战后的福祉作出贡献。1944年11月17日,时任美国总统罗斯福给战时科学研究发展局的主任万尼瓦尔·布什写了一封信,提出了4个问题。在信中,罗斯福总统要求布什就如何把战时的经验用于即将到来的和平时期的问题提出意见,特别是以下四个问题:

\begin{enumerate}[(1)]
    \item 在维护国家安全的限度内,为了把战时工作中科学知识所作的贡献尽快\textbf{公之于世},应该做些什么? 
    \item 特别是关于\textbf{科学向疾病}作斗争,国家如何组织研究,把战时取得的进展继续下去? \item 政府怎样促进和帮助\textbf{公立与私立机构}的研究活动? 
    \item 国家如何更好地发现和培养美国\textbf{青年人}的科学才能,以确保将来的科学研究水平及得上战争期间达到的水平?
\end{enumerate}

响应总统的要求,布什组织了全国50余位杰出的科学家和其他学者,组成四个委员会专门研究罗斯福总统提出的四个问题。8个月后,即1945年7月5日,布什把完成的研究报告呈交给杜鲁门总统--他是在罗斯福当年4月病逝后接任总统职务的。这一份报告由一个布什撰写的概述性文章和四个作为附件的分报告组成。布什赋予这份报告一个富有想象力的题目: 《科学--没有止境的前沿》 ,展现了科学的前景作为-“没有止境的边疆”的科学将会取代美国西部物理上的边疆,成为国家的经济发展、提高生活标准和推动社会变化的新的动力。

布什在给总统的呈文中写道: “我们国家中的开创精神仍然是朝气蓬勃的。开创者由完成他的任务的工具,科学则为他提供了广阔的尚未开发的内地。这种探索给予整个民族和个人的报酬是极大的。科学的进步是我们国家的安全、我们身体的更加健康、更多的就业机会、更高的生活水准以及文化进步的一个重要的关健。”\cite{ref1}

\section{基本观点与主要内容}

报告的中心思想是:
\begin{itemize}
    \item 国家安全、人民健康、公共福利需要新的科学知识,需要科学进步;
    \item 基础研究是一切知识的来源,是人民健康、公众福利和国家安全的源泉,保证科学的自主性和自由会带来繁荣和利益;
    \item 联邦政府应承担责任,对科学进行强有力的投资,促进产生新的科学知识,培训青年人的科学才能;
    \item 联邦政府建立一种维护科学自由探索(freedom of inquiry)保障长期稳定支持的新的机构--布什称之为国家科学基金会。
    
\end{itemize}
布什设想的国家研究基金会是一个全面包括自然科学各个领域的资助机构,包括生物学和医学,并且包含一个支持长期军事研究的部门,而且这个新机构具有协调整个国家科学技术发展的功能,这意味着它将成为白宫和国会科学政策的顾问。

《科学:没有止境的前沿》这份研究报告的基本论证,可以归结如下:“基础研究导致新知识。它提供科学资本。它创造储备,知识的实际应用必须从中提取。新产品和新工艺过程显得很不成熟,它们是建立在新的原理和新观念之上的,而新原理和新观念本身又是通过最纯粹的科学领域里的研究而艰苦地发展着。”\cite{ref1}

“有一条顽强的支配科学研究的规律:在要求取得立竿见影的成果的压力下,应用研究必然会排斥纯科学研究,除非制定
深思熟虑的政策以防止这种情况的出现”\cite{ref1}

“一个在新基础科学知识上依赖于其他国家的国家,它的工业进步将是缓慢的,它在世界贸易中的竞争地位将是虚弱的,不管它的机械技艺多么高明。”\cite{ref1}

“联邦政府应该接受新的职责,鼓励科学知识的创造和青年科学人才的培养。如果学院、大学和研究所要满足工业和政府对科学知识迅速增加的要求,那么,应该通过使用政府的资金来加强学院、大学和研究所的基础研究。”\cite{ref1}

布什在报告中建议设立国家研究基金会。“国家研究基金会应该发展和促进国家的科学研究和科学教育政策,应该资助非营利组织中的基础研究工作,应该通过奖学金和研究补助金来培养美国青年中的科学人才,应该靠合同和其他的方式支持军事问题的长期研究工作。”布什为国家研究基金会提出了这样的构想,它支持政府系统之外的研究机构从事科学研究。

\section{意义与影响}

该报告回应了罗斯福总统提出的有关美国战后科学发展的四大问题,扭转了美国科学发展“重应用,轻基础”的历史取向,它是战后美国科学发展的道路上的重大历史拐点,具有重大历史地位和深远启迪意义\cite{ref2}。可以从以下几个方面谈一下《科学:没有止境的前沿》的意义及其影响:

首先,布什的报告论证了\textbf{基础研究的重要性},论证了政府支持基础研究的\textbf{正当性},最终导致美国于1950年建立国家科学基金会。50多年来,美国联邦政府一直是支持基础研究的主力,它资助的基础研究经费占全国基础研究经费的比例,约在60\%。

第二,《科学:没有止境的前沿》提出了技术创新的\textbf{线性模式},成为美国科技政策的基本范式。这一模式强调科学研究和技术发明是推进技术创新的主要动力。科学研究主要由科学共同体进行,他们往往不考虑知识生产是否具有经济需求。企业家和企业则从科学共同体那里接手科技知识,开发科技知识的商业潜力,将之转化为新产品或新工艺,推向市场。在布什报告发表之后的20多年里,政府大力投资基础研究和高等教育,努力为工业创新打造知识资源和人力资源的基础;这一时期是战后经济扩张期,以电子、石油化工、原子能等领域创新为基础的“技术一经济范式”,正在形成之中。

第三,布什报告提出,把大学作为基础研究的\textbf{主力军}。近40年来,美国基础研究的约50\%是由大学完成的。美国从事基础研究最重要的机构是200家研究型大学。“首要的,正是在这些机构(大学)中,科学家可以工作在一个相对免于不利的惯例、偏见和商业需要的压力的环境中。它们提供了相当程度的个人思想自由令人满意的基础研究的进展,很少发生在通常的工业实验室中。有一些显著的例外情况,是真的。但即使是在那样的例子中,在对科学发展起着如此重要的自由方面,工业实验室无法与大学相比。”

\section{思考与总结}

时至2020年,在寻求国家复兴,民族富强的今天,我们国家对于高质量发展的诉求越发强烈,这一点可以体现在近日召开的十九届五中全会上,全会报告中发展一词总共出现了72次,不难看出国家对于发展的重视程度。要实现新发展,高质量发展,必然要求有更多高素质人才的出现。科技创新,
是提升一个国家核心实力的重要战略。

观察我国现在的科技发展,可以看出的是,我国在科技创新的投入上是越来越多,这一点可以看出国家与中央对于这件事情的大力支持。在论文发表数量上,这里以我学习的人工智能与计算机科学相关的领域为例,中国学者在人工智能顶级期刊和会议上发表的论文数量占据了相当巨大的比例。这说明我们国家是具备人才,
极其具备潜力的。但是又有学者指出,尽管我国的论文数量非常多,但是这些研究多偏向于应用领域,例如将人工智能技术应用在别的一个交叉领域(农业,工业检测,医疗等),这确实是一种很好的创新模式,但是对于人工智能算法本质的数学原理、系统软件的设计等方面,依然存在很大空缺。这些弱点,
我们带来了被别人“卡脖子”的可能。因此如果长期忽视下去,必然是巨大的隐患。

从万尼瓦尔·布什的这份报告中可以了解到,重视基础科学研究,对增强一个国家整体的科技实力,提升应用科学技术的高速发展,是有许多帮助的。
庆幸的是,我们也看到了许多中国的公司、企业、专家、学者等看到了这个问题,积极的投入到解决“卡脖子”的关键技术的研究之中,
例如华为公司研发的5G通信技术,寒武纪公司研发的深度学习处理器\cite{ref4},当然还有百度\cite{ref5}、旷视以及清华大学\cite{ref6}都自主开发的独具特色的深度学习
编程框架,这确实是令人开心的一件事情。

作为肩负民族复兴使命的青年人,又身处在这样一个充满机遇的时代,我也时刻能感受到不断学习知识、勇敢创新的重要性。在今年11月份,国家组织开展了全国科学道德和学风建设宣讲教育报告会,深入推进科学道德和学风建设宣讲教育工作,我有幸
报名在分会场观看了视频直播。这场报告会请到了敦煌研究院院长樊锦诗先生与戚发轫院士等知名科学家为我们分享与报告,在分享过程中,他们都不断强调了
科研严谨细致,求真务实的重要性,也多次提到,搞科学研究的过程中,必然会遇到各种各样的困难,鼓励我们坚持到底,迎难而上。这场报告会是党中央弘扬科学家精神,鼓励年轻人求真创新的重要举措,我希望能够以这些科学家为榜样,在今后的科研道路上,不负初心,求真务实,严谨创新,做出扎实的成果。

\nocite{*}

\begin{thebibliography}{10}
    \bibitem{ref1}布什. 科学: 没有止境的前沿[J]. 译. 北京: 商务印书馆, 2004.
    \bibitem{ref2}苏开源. 一份扭转乾坤的科学报告——《 科学——没有止境的前沿》 60 年述评[J]. 世界科学, 2005 (8): 41-42.
    \bibitem{ref3}李衍达. 再谈基础研究的重要性——读《 科学——没有止境的前沿》 报告有感[J]. 科技导报, 2014, 32(17): 1-1.
    \bibitem{ref4}Chen T, Du Z, Sun N, et al. Diannao: A small-footprint high-throughput accelerator for ubiquitous machine-learning[J]. ACM SIGARCH Computer Architecture News, 2014, 42(1): 269-284.
    \bibitem{ref5}Ma Y, Yu D, Wu T, et al. PaddlePaddle: An open-source deep learning platform from industrial practice[J]. Frontiers of Data and Domputing, 2019, 1(1): 105-115.
    \bibitem{ref6}Hu S M, Liang D, Yang G Y, et al. Jittor: a novel deep learning framework with meta-operators and unified graph execution[J]. Information Sciences, 2020, 63(222103): 1-222103.
\end{thebibliography}
